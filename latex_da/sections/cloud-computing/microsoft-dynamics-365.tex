Customer Relationship Management (CRM) beinhaltet eine Reihe von Software-Tools, die für das Verwalten der drei Pfeiler zwischen Kunde und Unternehmen zuständig sind: Marketing, Verkauf und Dienstleistungen. Neben Sammeln von Kudendaten aus verschiedenen Quellen und Automatisieren von sich wiederholenden Vertriebs-, Marketing- und Kundendienstprozessen, fördert eine CRM-Software auch die Abteilungsübergreifende Zusammenarbeit. Diese CRM-Software lässt sich in zwei Arten unterscheiden:
\begin{enumerate}
    \item On-Premise CRM-Software
    \item Cloud-basierte CRM-Software
\end{enumerate}

\subsubsection{On-Premise CRM-Software}
Da das Hauptgeschäft von Unternehmen wie Gesundheits- und Finanzeinrichtungen, der Umgang mit sensiblen Kundendaten ist, wird aus Sicherheitsgründen eine On-Premise Lösung, sprich eine CRM-Software vor Ort, bevorzugt. Diese Systeme sind aber mit hohen Wartungs- und Entwicklungskosten verbunden, und die Unternehmen sind selbst für Sicherheit und Datenpflege zuständig.  

\subsubsection{Cloud-basierte CRM-Software}
Unternehmen haben die Möglichkeit die CRM-Software und die damit verbundenen Aufwände, in die Cloud zu verlagern. Im Gegensatz du der On-Premise Lösung ist die Cloud-Variante flexibler in Hinsicht auf Speicher- und Rechenressourcen.