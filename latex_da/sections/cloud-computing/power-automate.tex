Power Automate ist ein Teil der Microsoft Power-Platform, mit der man Cloud-basierte Automatisierungsprozesse erstellen und verwalten kann. Der Benutzer hat Zugriff auf verschiedene Dienste von Microsoft und kann ohne oder wenig Code einen Workflow erstellen. Power Automate reagiert auf einen im Vorhinein definierten Trigger und führt eine Reihe von vordefinierten Schritten aus. Dieser Trigger kann eine neue E-Mail, ein neuer Datensatz im CRM eingefügt wird oder einfach die Erstellung eines neuen Dokuments im SharePoint sein. Somit ist es möglich auch ohne jegliche Vorkenntnisse oder fachspezifisches Wissen eine einfache Anwendung zu erstellen. Zum Beispiel kann beim Eingang einer E-Mail, die eine Kundenrechnung enthält, an den AI-Builder weitergegeben werden und dort mit einem vortrainierten Modell, die wichtigsten Daten herauslesen und dann erneut mit einer E-Mail versendet werden.

\subsection{Cloud Flow}
Da bei dieser Arbeit nur der Cloud Flow in Verwendung ist, wird hier nur auf diesen genauer eingegangen. Power Automate bietet hier drei verschiedene Arten von Cloud Flows an:

\begin{enumerate}
    \item Automated Flow
    \item Instant Flow
    \item Scheduled Flow
\end{enumerate}

\subsubsection{Automated Flow}
Dieser Flow wird ausgelöst, wenn eine gewünschte Kondition eintritt. Solche Trigger können, der Eingang einer E-Mail einer speziellen Person, eine Teams-Nachricht oder wenn ein Dokument in OneDrive geändert wird. Sogenannte Connectors werden verwendet, um eine Folge von Schritten zu bilden, um den gewünschten Effekt zu erzielen.

\subsubsection{Instant Flow}
Ein Instant Flow wird mit dem Klicken eines Buttons gestartet. Dieser Flow läuft sowohl auf Mobile als auch auf Desktop Devices. Er wird verwendet, um ein breites Spektrum von Aufgaben zu erledigen, wie die Beantragung einer Genehmigung via Teams oder SharePoint.

\subsubsection{Scheduled Flow}
Dieser Flow wird nach einem Zeitplan automatisiert und zu einem gewünschten Datum oder beliebiger Uhrzeit ausgeführt werden. Dieser Ablauf ist hilfreich, um zum Beispiel jeden Tag zur selben Uhrzeit einen Daten-Upload zu machen.

\subsubsection{Cloud Flow in der Praxis}
TODO!!!!!