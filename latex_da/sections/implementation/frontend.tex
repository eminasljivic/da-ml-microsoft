\subsection{Verwendete Technologien}

\subsubsection{Angular}

Angular ist ein auf TypeScript basierendes Frontend-Framework, welches neben React, Express und Vue.js das beliebteste Framework auf dem Markt ist. \cite{SFT} Dabei wird Angular sowohl für kleine Projekte in der Schule, als auch große Projekte in weltbekannten Plattformen verwendet.

\paragraph*{Components}

Das Frontend wird in Komponenten (Components) aufgeteilt und stellen unterschiedliche Elemente der Darstellung dar, welche auch mehrmals untereinander verwendet werden können, um repetitiven Code zu verhindern.

Jedes Component besteht aus

\begin{itemize}
    \item einer Component-Klasse (TypeScript-File),
    \item einem HTML-Template und
    \item einem Stylesheet (CSS/SCSS-File).
\end{itemize}

\begin{lstlisting}[language=TypeScript,caption={Beispiel eines Angular-Components}]
    @Component({
        selector: 'custom-app',
        templateUrl: './app.component.html',
        styleUrls: ['./app.component.scss']
    })
    export class AppComponent {

    title: string = "Meine eigene App!";

    items: any[] = [
        {
            "text": "Das ist meine"
        },
        {
            "text": "erste eigene Angular-App."
        }
    ];

    // Restlicher Code
    }
\end{lstlisting}

Wie man im obigen Beispiel sieht müssen bei einem Component Felder gesetzt werden, damit die unterschiedlichen Files miteinander assoziierbar sind:

\begin{table}[H]
    \centering
    \begin{tabular}{|l|l|p{0.5\linewidth}|}
        \hline
        selector             & verpflichtend & Gibt den HTML-Tag an, mit welchem man das Component in anderen aufrufen kann (Bsp.: <app-custom-app></app-custom-app>)       \\ \hline
        template/templateUrl & verpflichtend & Gibt entweder direkt den HTML-Code (template) an oder referenziert zu einem HTML-File (templateUrl)                          \\ \hline
        styles/styleUrls     & freiwillig    & Gibt entweder direkt einen oder mehrere CSS-Codes (styles) an oder referenziert zu einem oder mehreren CSS-Files (styleUrls) \\ \hline
    \end{tabular}
\end{table}

\paragraph{HTML-Templates}

Neben der generellen Struktur des Frontends, kann ein HTML-Template auch Daten aus dem Component anzeigen.

\begin{lstlisting}[language=HTML, title="Beispiel für ein HTML-Template"]
    <h1>{{title}}</h1>

    <ul>
        <li *ngFor="let item of items">{{item.text}}</li>
    </ul>
\end{lstlisting}

Mithilfe von unterschiedlichen Instuktionen


\paragraph*{CLI}

\subsubsection{HTML}
\subsubsection{CSS/SCCS}



\subsubsection{TypeScript}