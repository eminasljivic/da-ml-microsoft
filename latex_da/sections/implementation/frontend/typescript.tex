\subsubsection{TypeScript}

Laut der jährlichen Entwicklerumfrage von StackOverflow landet TypeScript nach Python auf dem zweiten Platz (mit 15,29\%) der meistgesuchten Programmiersprachen. Damit liegt TypeScript mit 0,7\% vor JavaScript, die Sprache auf der TypeScript basiert. \cite{SFT}

TypeScript versucht JavaScript zu verbessern, indem es eine strikte Typisierung fordert. Damit beugt man dem einfachen Fehler vor, dass Werte einen anderen Typen enthalten, als der gefordert ist. Dies führt dazu, dass Typenfehler bereits in der Entwicklung gefunden werden und verhindert, dass diese Fehler überhaupt in der Produktivumgebung aufkommen.

Abgesehen davon sind die zwei Sprachen ident, daher ist es auch möglich bereits bestehende Tools von JavaScript mit TypeScript zu verwenden. Das heißt, dass alle JavaScript-Libraries weiterhin genutzt werden können. Die beliebtesten Libraries sind zum Beispiel jQuery und Chart.js.
