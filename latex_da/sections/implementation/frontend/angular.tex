\subsubsection{Angular}

Angular ist ein auf TypeScript basierendes Frontend-Framework, welches neben React, Express und Vue.js das beliebteste Framework auf dem Markt ist. \cite{SFT} Dabei wird Angular sowohl für kleine Projekte in der Schule, als auch große Projekte in weltbekannten Plattformen verwendet.

\paragraph*{Components}

Das Frontend wird in Komponenten (Components) aufgeteilt und stellen unterschiedliche Elemente der Darstellung dar, welche auch mehrmals untereinander verwendet werden können, um repetitiven Code zu verhindern.

Jedes Component besteht aus

\begin{itemize}
    \item einer Component-Klasse (TypeScript-File),
    \item einem HTML-Template und
    \item einem Stylesheet (CSS/SCSS-File).
\end{itemize}

\begin{lstlisting}[language=TypeScript,caption={Beispiel eines Angular-Components}]
    @Component({
        selector: 'custom-app',
        templateUrl: './app.component.html',
        styleUrls: ['./app.component.scss']
    })
    export class AppComponent {

    title: string = "Meine eigene App!";

    items: any[] = [
        {
            "text": "Das ist meine"
        },
        {
            "text": "erste eigene Angular-App."
        }
    ];

    // Restlicher Code
    }
\end{lstlisting}

Wie man im obigen Beispiel sieht müssen bei einem Component Felder gesetzt werden, damit die unterschiedlichen Files miteinander assoziierbar sind:

\begin{table}[H]
    \centering
    \begin{tabular}{|l|l|p{0.5\linewidth}|}
        \hline
        selector             & verpflichtend & Gibt den HTML-Tag an, mit welchem man das Component in anderen aufrufen kann (Bsp.: <app-custom-app></app-custom-app>)       \\ \hline
        template/templateUrl & verpflichtend & Gibt entweder direkt den HTML-Code (template) an oder referenziert zu einem HTML-File (templateUrl)                          \\ \hline
        styles/styleUrls     & freiwillig    & Gibt entweder direkt einen oder mehrere CSS-Codes (styles) an oder referenziert zu einem oder mehreren CSS-Files (styleUrls) \\ \hline
    \end{tabular}
    \caption{Parameter einer Komponente}
\end{table}

\paragraph{HTML-Templates}

Neben der generellen Struktur des Frontends, kann ein HTML-Template auch Daten aus dem Component anzeigen.

\lstinputlisting[language=HTML, caption={Beispiel fuer ein HTML-Template}]{code-snippets/first-app/src/app/app.component.html}

Generell besteht die Struktur aus üblichen HTML-Tags, jedoch kann es zusätzlich noch Component-Tags oder auch programmierähnliche Operationen beinhalten. Hierbei unterscheidet man zwischen:

\begin{itemize}
    \item Interpolation
    \item Property Binding
    \item Event Binding
    \item Two-Way Binding
\end{itemize}


\subparagraph{Interpolation} Bei einer Interpolation werden zwei geschwungene Klammern ''\{\{\}\}'' genutzt, um den Wert einer Variable im Frontend anzuzeigen. Dabei muss darauf geachtet werden, dass die Interpolation immer den String-Wert der Variable nutzt. \cite{ANG-IN}

\subparagraph{Property Binding} Mit einem Property Binding können genau wie bei der Interpolation Daten im Frontend angegeben werden, jedoch kann jeder Datentyp genutzt werden.

\subparagraph*{Event Binding} Damit kann das Backend auf gewisse Aktionen (Klicken, Bewegung der Maus, Hover) im Frontend reagieren.

\subparagraph*{Two-Way Binding} Die Mischung aus  den vorherigen ergibt das Two-Way Binding. Sowohl das Backend als auch das Frontend reagiert auf Änderungen der jeweils anderen Komponente.

\paragraph*{CLI}

Eine \gls{cli} ermöglicht das Nutzen von Funktion über die Kommandozeile, spezifisch für Angular wird darüber das Erstellen, Generieren und Verwalten gesteuert. Das Kommandowort für Angular lautet \lstinline{ng} und darauf können unterschiedliche Funktionswörter folgen.