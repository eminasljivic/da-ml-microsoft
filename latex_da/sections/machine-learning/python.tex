\subsection{Python als Progrmmiersprache}

Nach dem Scheitern seiner ersten Programmiersprache, entwickelte Guido van Rossum die Sprache Python, dabei wollte er alle Fehler, die er beim Entwickeln von ABC gefunden hat, verbessern. Daher basieren auch die Stukturen und Konventionen von Python auf Unix, ohne an Unix gebunden zu sein. 

Python unterscheidet sich in vielen Punkten zu anderen Sprachen, unteranderem dass sie viel Wert auf Lesbarkeit gibt. Die auffälliste davon ist, dass Einrückungen Codeblöcke unterteilt anstatt eine Art von Klammern. Dafür gibt es zwei Gründe:

\begin{itemize}
    \item Es macht den Code kürzer und er wirkt nicht unnötig ausgeschmückt, daher braucht man eine kürzere Aufmerksamkeitsspanne um den Sinn einer Codestelle nachvollziehen zu können.
    \item Die Stuktur des Codes ist vereinhaltlicht, was es einfacher macht Projekte von anderen zu verstehen.
\end{itemize}

Außerdem werden dem Entwickler in vielen Entscheidungen leichter gemacht, da unnötige Möglichkeiten entfernt wurden, das heißt, dass es meisten nur eine offensichtliche Art und Weise gibt, etwas zu implementieren. Dazu kommt noch die Nutzung von Spezialzeichen, es werden nur Zeichen unterstützt, die den meisten bereits bekannt sind und dessen Operation Sinn machen. \cite{PythonGVR:online}

Dies beantwortet jedoch nicht die Frage ''Wieso ist Python die beliebteste Sprache für \gls{ml}?''. Die Antwort darauf ist, dass die Sprache nicht nur triviale Aufgaben bereits vorimplementiert, sonder auch, dass die meisten ML Funktionen in Python Libraries zusammengefasst sind. Daher muss man als Neuanfänger oder Fortgeschrittener keine bereits gelösten Probleme von Grund auf noch einmal angehen.

\subsection{Notebooks}

Da es bei \gls{ml} es öfters dazu kommt, dass bestimmte Codeblöcke oft wiederholt ausgeführt werden, kann man mit Hilfe von Notebooks diese einzelen ausführen. Zum Beispiel beim Analysieren eines Datensatzes oder schnell kleine Änderung am geplanten Vorgehen vorzunehemen.

Diese Codeblöcke können entweder Code, Texte oder Grafiken beinhalten, diese werden fortlaufend mit einer Nummer versehen und die Ausgabe/Grafik erscheint direkt unter dem Code.

\begin{figure}
    \centering
    \includegraphics{}
\end{figure}