\subsubsection{CSS}

Um Webseiten zu gestalten, wird \gls{css} verwendet, damit kann das Design von der generellen Struktur und dem Inhalt getrennt werden. Wobei \gls{css} immer mit \gls{html} assoziiert wird, kann es mit jeder auf \gls{xml} basierenden Ausdruckssprache benutzt werden. \cite{HTML-CSS}

Generell besteht eine \gls{css}-Regel aus einem Key/Value-Pair, welches die Styling-Property (was soll geändert werden) und den Styling-Wert (auf was soll es geändert werden) setzt. Dabei wird der Styling-Wert je nach Styling-Property eingeschränkt.

Zusätzlich können mehrere \gls{css}-Regeln zu einer zusammengefasst werden, diese Regeln werden als Shorthand bezeichnet (z.B.: \lstinline{background}, \lstinline{border}). Abhängig von der Werteanzahl wird ein bestimmtes Shorthand referenziert. Die Styling-Property \lstinline{margin} kann folgendermaßen angegeben werden:

\begin{itemize}
    \item Ein Wert: \lstinline{margin: 10px}

          Dieser Wert wird für alle Seiten genutzt.
    \item Zwei Werte: \lstinline{margin: 10px 5px}

          Der erste Wert wird für die obere und untere Seite genutzt und der zweite für die linke und rechte.
    \item Drei Werte:  \lstinline{margin: 10px 5px 15px}

          Der erste Wert wird für die obere Seite genutzt, der zweite Wert für die linke und rechte Seite und der letzte Werte für die untere.
    \item Vier Werte: \lstinline{margin: 10px 5px 15px 20px}

          Die Werte werden im Uhrzeigersinn beginnend mit der oberen Seite vergeben.
\end{itemize}

Beispiele für gültige Properties:

\begin{table}[H]
    \centering
    \begin{tabular}{|l|l|p{0.5\linewidth}|}
        \hline
        \lstinline|width/height|  & \lstinline|100px| / \lstinline|100%| & Gibt die Breite/Höhe eines HTML-Elements an (als Länge oder Prozentzahl) \\ \hline
        \lstinline|display|  & \lstinline|block| / \lstinline|flex| & Gibt an wie die HTML-Elemente angezeigt werden                           \\ \hline
        \lstinline|margin/padding|  & \lstinline|10px |                          & Gibt den inneren/äußeren Abstand an                                      \\ \hline
        \lstinline|background| & \lstinline|#fff|                         & Gibt die Farbe/das Bild für den Hintergrund an                           \\ \hline
        \lstinline|color| & \lstinline|red|                         & Gibt die Farbe/Strichstärke an                                           \\ \hline
    \end{tabular}
    \caption{Beispiele für CSS-Properties}
\end{table}

Um die Struktur mit dem Styling zu verbinden, bietet \gls{html} drei Möglichkeiten an: \cite{CSS}

\begin{itemize}
    \item Direkt im \gls{html}-File innerhalb vom \lstinline{<head>}-Tag, mithilfe von einem \lstinline{<style>}-Tag.

          \begin{lstlisting}[language=HTML,numbers=none]
<head>
    <style>
        <!-- CSS-Regeln -->
    </style>
</head>
    \end{lstlisting}
    \item Als Referenz zu einem separatem \gls{css}-File (Dateinamenserweiterung: \lstinline{.css}) wird ein \lstinline{<link>}-Tag im \lstinline{<head>}-Tag gesetzt. Wobei diese Möglichkeit es ermöglicht, die \gls{css}-Regeln in mehreren \gls{html}-Files zu nutzen und man damit Codeverdoppelung vermeidet.

          \begin{lstlisting}[language=HTML,numbers=none]
<head>
    <link rel="stylesheet" href="pfad/zum/css-file.css">
</head>
    \end{lstlisting}
    \item Direkt im \gls{html}-Tag mit dem \lstinline{style} Attribut.

          \begin{lstlisting}[language=HTML,numbers=none]
<div style="<!-- CSS-Regeln -->">
            \end{lstlisting}
\end{itemize}

Um eine \gls{css}-Regel zuweisen zu können, werden sogenannte Selektoren angegeben, welche direkt ein \gls{html}-Element ansprechen oder mehrere zusammenfassen. Gültige Selektoren sind:

\begin{table}[H]
    \centering
    \begin{tabular}{|l|l|p{0.7\linewidth}|}
        \hline
        \lstinline|*| & \lstinline|*| & Selektiert alle HTML-Elemente                                                      \\ \hline
        \lstinline|element| & \lstinline|h1| & Selektiert alle HTML-Elemente mit dem angegebenen Elementnamen                     \\ \hline
        \lstinline|.klasse| & \lstinline|.nav| & Selektiert alle HTML-Elemente mit dem angegebenen \lstinline|class|-Attribut \\ \hline
        \lstinline|#id| & \lstinline|#main| & Selektiert alle HTML-Elemente mit dem angegebenen \lstinline|id|-Attribut \\ \hline
    \end{tabular}
    \caption{CSS-Selektoren}
\end{table}

Um diese Spezifikation noch genauer zu definieren, können Selektoren mit unterschiedlichen Operatoren zusammengefügt werden.

\begin{table}[H]
    \centering
    \begin{tabular}{|l|l|p{0.4\linewidth}|}
        \hline
        \lstinline|,|      & \lstinline|h1, p| & Selektiert alle \lstinline|<p>| und \lstinline|<h1>|-Elemente                                \\ \hline
        '' '' (Leerzeichen)          & \lstinline|div p| & Selektiert alle \lstinline|<p>|-Elemente in einem \lstinline|<div>|-Elemente                  \\ \hline
        \lstinline|>|      & \lstinline|div > p| & Selektiert alle \lstinline|<p>|-Elemente, die direkt in einem \lstinline|<div>|-Elemente sind \\ \hline
        (2 Selektoren ohne Operator) & \lstinline|div.main| & Selektiert alle \lstinline|<div>|-Element, die auch die Klasse \lstinline|.main| haben          \\ \hline
    \end{tabular}
    \caption{CSS-Operatoren}
\end{table}

\paragraph{SCSS}

Da die Funktionalitäten von \gls{css} sehr eingeschränkt sind, wird oftmals als Weiterentwicklung \gls{scss} genutzt. Dies ermöglicht das Vereinfachen von Regeln und macht diese generell besser überschaubar.

Zusätzlich zu den normalen Funktionen von \gls{css} stellt \gls{scss} zum Beispiel folgende zur Verfügung: \cite{SCSS}

\begin{itemize}
    \item Variablen

          Um über mehrere Regeln einen gleichen Wert zu setzen, kann dieser Wert in einer Variable gespeichert werden. Dies ermöglicht ein leichtes Updaten an mehreren Stellen gleichzeitig.

    \item Nesting

          Da mit der Zeit die Selektoren sehr kompliziert werden, ermöglicht \gls{scss} das Erstellen von Regeln in einer hierarchischen Abfolge.

    \item Modules

          Damit man Codeverdoppelung vermeidet, kann man weitere \gls{css}-Files importieren.
\end{itemize}