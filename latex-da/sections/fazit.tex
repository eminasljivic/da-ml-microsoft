\setauthor{Emina Sljivic}

\section{Gelerntes}

Dadurch, dass diese Diplomarbeit mein erster Kontakt mit OCR, NER und generell Informationsextraktion war, wurde die meiste Zeit in die Recherche gesteckt. Während der Recherche bin ich auf unterschiedliche Ansätze gestoßen, davon wurden die meisten auch ausgetestet und gleich wieder verworfen, da sie für unseren Anwendungsfall nicht geeignet sind. 

Ein Beispiel dafür war die Tabellenextraktion, bei der drei unterschiedliche Ansätze getestet wurden. Als Erstes wurden Libraries getestet, welche von sich selber die Möglichkeit anbieten, eine ganze Methode zu extrahieren. Jedoch waren diese nicht passend, da unsere Beispieldaten nicht herkömmlich formatiert sind. Das gleiche Problem war der Grund, dass die komplett selbst entwickelte Tabellenextraktion verfallen wurde. 

Außerdem wurde beim Antrainieren des Modells getestet, was die minimale Anzahl der Trainingsdaten ist. Daher wurde das erste Modell mit nur fünf Rechnungen trainiert, wo auch schnell klar wurde, dass dies viel zu wenig sind, da zum Beispiel die Adressen nur zum Teil extrahiert wurden und manche Feld komplett falsch kategorisiert wurden. Daraufhin wurde die Anzahl der Trainingsdaten stufenhaft erhöht und am Ende wurde kam die optimale Anzahl auf rund 15 Rechnungen.

Abgesehen vom technischen Teil wurde mir auch organisatorisch viel für die Zukunft mitgegeben. Da ich im Vorfeld nicht damit gerechnet habe, dass man mehrere unterschiedliche Ansätze austesten muss, bis man den richtigen findet, wurde viel mehr Zeit in die Entwicklung gesteckt, als am Anfang geplant.

\section{Danksagung}

An dieser Stelle möchte ich mich bei all denjenigen bedanken, die mich während der Anfertigung dieser Masterarbeit unterstützt und motiviert haben.

Ich bedanke mich bei jedem Mitarbeiter von smartpoint, die mich in jeglicher Art unterstützt haben. Zusätzlich bedanke ich mich bei smartpoint dafür, dass sie für die Diplomarbeit eine Testumgebung zur Verfügung gestellt haben und dass sie auch nach der Praktikumszeit für Fragen ein offenes Ohr hatten. 

Außerdem möchte ich mich bei Nico Bojer für das mehrmalige Korrekturlesen meiner Diplomarbeit danken. 

Weiters gilt mein Dank meinem Diplomarbeitsbetreuer, Karpowicz Michał, welcher mich auch in stressigen Situationen unterstützt hat.