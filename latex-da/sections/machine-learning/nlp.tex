\subsection{Natural Language Processing}

\gls{nlp} gehört zu den Disziplinen der künstlichen Intelligenz, welche konstant überarbeitet und verbessert werden. Daher wird der Begriff \gls{nlp} auch oft neuerfunden und anders interpretiert, dabei liegt der große Fokus auf das Wort ''Processing'' oder auf Deutsch ''Verarbeitung''. Das eigentliche Ziel von \gls{nlp} liegt darin, eine für Menschen verständliche Sprache zu verstehen und damit umgehen zu können. Jedoch scheitert es daran, dass eine \gls{ki} der heutigen Zeit nicht im Stande dazu ist, eine Schlussfolgerung aus einem Text zu ziehen. \cite{NLP}

Die Texte reichen von täglichen E-Mails bis zu wissenschaftlichen Texten und sich aus dem Alltag nicht mehr wegdenkbar. Zu diesen Aufgabenbereichen gehören zum Beispiel: \cite{NLP-UC}

\begin{itemize}
    \item Spam Erkennung
    
    Ohne \gls{nlp} wären alle E-Mail-Postfächer voll mit Spammails. Dabei schaut ein \gls{nlp}-Modell auf die Grammatik, gewisse Begriffe und falsch geschriebene Wörter, da Spam-Emails oft die gleichen Charakteristiken von frei übersetzten Texten aufweisen. 

    \item Maschinen Übersetzungen
    
    Da der Sinn eines Satzes nicht verloren gehen soll, wird beim Übersetzen \gls{nlp} genutzt, um den Kontext zu übernehmen. 
\end{itemize}

Darüber hinaus wird \gls{nlp} in verschiedene Bereiche unterteilt.