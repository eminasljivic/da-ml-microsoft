\section{Deep Learning}

Sowie beim \gls{ml} sind \gls{dl}-Algorithmen abhängig von antrainierten Daten und kann daher als Synonym oder Untergruppe gesehen werden. Der grobe Unterschied liegt darin, dass diese Daten meist komplett roh sind und keine Vorarbeit geleistet wurde. Daher kann ein \gls{dl}-Modell wie ein Mensch noch nie davor gesehene, spezielle Bilder kategorisieren. 

Die Einsatzmöglichkeiten überlappen sich sehr mit jenen von \gls{ml}, jedoch ist das Resultat von \gls{dl}-Modellen viel präziser. Daher würden die meisten modernen Assistenten ohne \gls{dl} auch nicht auf dem erwarteten Niveau arbeiten. 

Obwohl schon viele Anwendungen auf \Gls{dl} basieren, handelt es sich dabei um eine junge Technologie, die noch weit von ihrem vollen Potenzial entfernt ist.
